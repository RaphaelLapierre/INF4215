%%%%%%%%%%%%%%%%%%%%%%%%%%%%%%%%%%%%%%%%%
% Thin Sectioned Essay
% LaTeX Template
% Version 1.0 (3/8/13)
%
% This template has been downloaded from:
% http://www.LaTeXTemplates.com
%
% Original Author:
% Nicolas Diaz (nsdiaz@uc.cl) with extensive modifications by:
% Vel (vel@latextemplates.com)
%
% License:
% CC BY-NC-SA 3.0 (http://creativecommons.org/licenses/by-nc-sa/3.0/)
%
%%%%%%%%%%%%%%%%%%%%%%%%%%%%%%%%%%%%%%%%%

%----------------------------------------------------------------------------------------
%	PACKAGES AND OTHER DOCUMENT CONFIGURATIONS
%----------------------------------------------------------------------------------------

\documentclass[a4paper, 12pt]{article} % Font size (can be 10pt, 11pt or 12pt) and paper size (remove a4paper for US letter paper)

\usepackage[protrusion=true,expansion=true]{microtype} % Better typography
\usepackage{graphicx} % Required for including pictures
\usepackage[utf8]{inputenc}
\usepackage[margin=1.0in]{geometry}
\usepackage{url}
\usepackage{fancyhdr}
\usepackage{amsmath}
\usepackage{setspace}
\usepackage{enumitem}
\setlength\parindent{0pt} % Removes all indentation from paragraphs

\usepackage[T1]{fontenc} % Required for accented characters
\usepackage{times} % Use the Palatino font

\usepackage{listings}
\usepackage{color}
\lstset{mathescape}

\definecolor{dkgreen}{rgb}{0,0.6,0}
\definecolor{gray}{rgb}{0.5,0.5,0.5}
\definecolor{mauve}{rgb}{0.58,0,0.82}

\lstset{frame=tb,
   language=python,
   aboveskip=3mm,
   belowskip=3mm,
   showstringspaces=false,
   columns=flexible,
   basicstyle={\small\ttfamily},
   numbers=none,
   numberstyle=\tiny\color{gray},
   keywordstyle=\color{blue},
   commentstyle=\color{dkgreen},
   stringstyle=\color{mauve},
   breaklines=true,
   breakatwhitespace=true
   tabsize=4
}
\linespread{1.00} % Change line spacing here, Palatino benefits from a slight increase by default

\makeatletter
\renewcommand{\@listI}{\itemsep=0pt} % Reduce the space between items in the itemize and enumerate environments and the bibliography

\renewcommand{\maketitle}{ % Customize the title - do not edit title and author name here, see the TITLE block below
\begin{center} % Right align

\vspace*{25pt} % Some vertical space between the title and author name
{\LARGE\@title} % Increase the font size of the title

\vspace{125pt} % Some vertical space between the title and author name

{\large\@author} % Author name

\vspace{125pt} % Some vertical space between the author block and abstract
Dans le cadre du cours
\\INF4215 - Introduction à l'intelligence artificielle
\vspace{125pt} % Some vertical space between the author block and abstract
\\\@date % Date
\vspace{125pt} % Some vertical space between the author block and abstract

\end{center}
}

%----------------------------------------------------------------------------------------
%	TITLE
%----------------------------------------------------------------------------------------

\title{Travail pratique \#1} 

\author{\textsc{Raphael Lapierre 1644671\\
	        Alexandre St-Onge Matricule 1623576} % Author
\vspace{10pt}
\\{\textit{École Polytechnique de Montréal}}} % Institution

\date{14 Février 2016} % Date
%----------------------------------------------------------------------------------------

\begin{document}

\thispagestyle{empty}
\clearpage\maketitle % Print the title section
\pagebreak[4]
%----------------------------------------------------------------------------------------
%	En tête et pieds de page 
%----------------------------------------------------------------------------------------

\setlength{\headheight}{15.0pt}
\pagestyle{fancy}
\fancyhead[L]{INF4215}
\fancyhead[C]{}
\fancyhead[R]{Travail pratique \#1}
\fancyfoot[C]{\textbf{page \thepage}}

%----------------------------------------------------------------------------------------
%	ESSAY BODY
%----------------------------------------------------------------------------------------
\section*{Explication des algorithmes}

\subsection*{Recherche en arbre}
La recherche en arbre commence avec l'état initial vide et \emph{n} enfants avec chacun 1 antenne qui couvre une position, où \emph{n} est le nombre de place à couvrir.
Par la suite, l'enfant d'un état est calculé en trouvant le point non couvert le plus près de la dernière antenne placé et en couvrant ce dernier soit en ajoutant une nouvelle 
antenne ou en aggrandissant la dernière antenne placé. L'algorithme de Djikstra est utilisé pour parcourir l'arbre d'état afin de trouver une solution.

\subsection*{Recherche locale}
Notre algorithme de recherche locale effectue un recuit simulé à partir d'une solution obtenue avec une algorithme vorace naïf. 
Le recuit simulé fonctionne en échangeant des points entre les antennes.

\section*{Questions}

\subsection*{Question 1}
Expliquez ce que fait cette fonction et fournissez un exemple utilisant cette fonction.
\begin{lstlisting}
def fct(predList, inputList):
	return filter(lambda x: all([f(x) for f in predList]), inputList)
\end{lstlisting}

La fonction filtre la liste \texttt{inputList} pour ne retourner que les éléments qui respectent tout les prédicats présents dans \texttt{predList}.
Si \texttt{inputList} est une liste d'état, il serait donc possible d'utiliser cette fonction pour filtrer la liste afin d'obtenir seulement que les états 
respectant tout nos prédicats.

\subsection*{Question 2}
\subsubsection*{Recherche en arbre}
Le point fort de la recherche en arbre que nous avons implémentée est son bon compromis entre une solution
acceptable et le temps de calcul. En effet, celui-ci ignore beaucoup d'états lorsqu'il ajoute les enfants à la frontière.
Cela a pour effet de ne pas garantir une solution optimale mais au moins le résultat peut arriver rapidement.
Le point faible de notre algorithme est par contre qu'il ne garanti justement pas une solution idéale. Cela dépend
des applications mais il se peut que ce point soit inacceptable.

\subsubsection*{Recherche locale}
En utilisant la méthode du recuit simulé, nous pouvons nous échapper d'un minimum local ce qui peut être avantageux
dans la recherche d'une bonne solution. Un des points faibles de notre implémentation est que notre solution de départ
donnée par un algorithme vorace est très naïve ce qui peut donner du fil à retordre au recuit simulé pour converger.
Finalement, les variables de température et de temps du recuit simulé ne sont pas tout à fait optimisées au problème.

\section*{Compétition}
Utiliser la recherche en arbre pour la compétition.

%----------------------------------------------------------------------------------------
\end{document}

