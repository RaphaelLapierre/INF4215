%%%%%%%%%%%%%%%%%%%%%%%%%%%%%%%%%%%%%%%%%
% Thin Sectioned Essay
% LaTeX Template
% Version 1.0 (3/8/13)
%
% This template has been downloaded from:
% http://www.LaTeXTemplates.com
%
% Original Author:
% Nicolas Diaz (nsdiaz@uc.cl) with extensive modifications by:
% Vel (vel@latextemplates.com)
%
% License:
% CC BY-NC-SA 3.0 (http://creativecommons.org/licenses/by-nc-sa/3.0/)
%
%%%%%%%%%%%%%%%%%%%%%%%%%%%%%%%%%%%%%%%%%

%----------------------------------------------------------------------------------------
%	PACKAGES AND OTHER DOCUMENT CONFIGURATIONS
%----------------------------------------------------------------------------------------

\documentclass[a4paper, 12pt]{article} % Font size (can be 10pt, 11pt or 12pt) and paper size (remove a4paper for US letter paper)

\usepackage[protrusion=true,expansion=true]{microtype} % Better typography
\usepackage{graphicx} % Required for including pictures
\usepackage[utf8]{inputenc}
\usepackage[margin=1.0in]{geometry}
\usepackage{url}
\usepackage{fancyhdr}
\usepackage{amsmath}
\usepackage{setspace}
\usepackage{enumitem}
\setlength\parindent{0pt} % Removes all indentation from paragraphs

\usepackage[T1]{fontenc} % Required for accented characters
\usepackage{times} % Use the Palatino font

\usepackage{listings}
\usepackage{color}
\lstset{mathescape}

\definecolor{dkgreen}{rgb}{0,0.6,0}
\definecolor{gray}{rgb}{0.5,0.5,0.5}
\definecolor{mauve}{rgb}{0.58,0,0.82}

\lstset{frame=tb,
   language=python,
   aboveskip=3mm,
   belowskip=3mm,
   showstringspaces=false,
   columns=flexible,
   basicstyle={\small\ttfamily},
   numbers=none,
   numberstyle=\tiny\color{gray},
   keywordstyle=\color{blue},
   commentstyle=\color{dkgreen},
   stringstyle=\color{mauve},
   breaklines=true,
   breakatwhitespace=true
   tabsize=4
}
\linespread{1.00} % Change line spacing here, Palatino benefits from a slight increase by default

\makeatletter
\renewcommand{\@listI}{\itemsep=0pt} % Reduce the space between items in the itemize and enumerate environments and the bibliography

\renewcommand{\maketitle}{ % Customize the title - do not edit title and author name here, see the TITLE block below
\begin{center} % Right align

\vspace*{25pt} % Some vertical space between the title and author name
{\LARGE\@title} % Increase the font size of the title

\vspace{125pt} % Some vertical space between the title and author name

{\large\@author} % Author name

\vspace{125pt} % Some vertical space between the author block and abstract
Dans le cadre du cours
\\INF4215 - Introduction à l'intelligence artificielle
\vspace{125pt} % Some vertical space between the author block and abstract
\\\@date % Date
\vspace{125pt} % Some vertical space between the author block and abstract

\end{center}
}

%----------------------------------------------------------------------------------------
%	TITLE
%----------------------------------------------------------------------------------------

\title{Travail pratique \#2} 

\author{\textsc{Raphael Lapierre 1644671\\
	        Alexandre St-Onge Matricule 1623576} % Author
\vspace{10pt}
\\{\textit{École Polytechnique de Montréal}}} % Institution

\date{20 Mars 2016} % Date
%----------------------------------------------------------------------------------------

\begin{document}

\thispagestyle{empty}
\clearpage\maketitle % Print the title section
\pagebreak[4]
%----------------------------------------------------------------------------------------
%	En tête et pieds de page 
%----------------------------------------------------------------------------------------

\setlength{\headheight}{15.0pt}
\pagestyle{fancy}
\fancyhead[L]{INF4215}
\fancyhead[C]{}
\fancyhead[R]{Travail pratique \#2}
\fancyfoot[C]{\textbf{page \thepage}}

%----------------------------------------------------------------------------------------
%	ESSAY BODY
%----------------------------------------------------------------------------------------
\section*{Questions}
\subsection*{Question 1}
Si l'on demande au code aime(paulo, X). celui-ci arrêtera de s'éxécuter après avoir utilisé
le chainage arrière avec mama\_burger(b) et retournera faux.

\subsection*{Question 2}
Des décisions probabilistes, ou tout simplement une requête qui ne peut s'exprimer avec des 
clauses de horn. 

\section{Réflexion}
La base de connaissance implémentés se base principalement sur des propriétés attribuées aux
cours à l'aide de la règle \textit{prop} ainsi qu'à la récursivité. Le code développé pourrait
utiliser plus de liste plutôt que d'énumérer toute les propriétés et les cours un à un. Par contre,
il semble étrange d'utiliser le Prolog pour faire quelque chose qui, nous semblerait plus simple à 
représenter par une base de donnés relationelle.

%----------------------------------------------------------------------------------------
\end{document}

